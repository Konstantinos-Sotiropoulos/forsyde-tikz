%%%%%%%%%%%%%%%%
% ENVIRONMENTS %
%%%%%%%%%%%%%%%%
\newif\ifnolabel
\newif\ifnocolor
\pgfkeys{
	/tikz/nomoccolor/.is if=nocolor,
	/tikz/nomoclabel/.is if=nolabel,
	/tikz/type style/.store in = \typeStyle,
	/tikz/type style = \scriptsize\textit,
	/tikz/label style/.store in = \labelStyle,
	/tikz/label style = \textbf,
	/tikz/function style/.store in = \funcStyle,
	/tikz/function style = \scriptsize,
	/tikz/moc/.store in = \MoC,
	/tikz/moc = none,
	/tikz/mocin/.store in = \MoCin,
	/tikz/mocin = none,
	/tikz/mocout/.store in = \MoCout,
	/tikz/mocout = none,
	/tikz/token pos/.store in = \tokPos,
	/tikz/token pos = 0.5,
	/tikz/deviate/.store in = \pDeviate,
	/tikz/deviate = 0pt,
}

\newcommand{\getmoclabel}[1]{\ifthenelse{\equal{#1}{sy}}{SY}{\ifthenelse{\equal{#1}{de}}{DE}{\ifthenelse{\equal{#1}{ct}}{CT}{\ifthenelse{\equal{#1}{sdf}}{SDF}{}}}}}

%%%%%%%%%%%%%%%%%
% GENERIC NODES %
%%%%%%%%%%%%%%%%%

% Applicative
\pgfkeys{
	/applicative/.is family, /applicative,
	default/.style = {moc=none, reverse = false, type= , ni=1, no=1, nf=0, f1=$ f_1 $, f2=$ f_2 $, f3=$ f_3 $, f4=$ f_4 $, line sep=0pt},
 	moc/.estore in = \pMoc,
 	type/.estore in = \pType,
	ni/.estore in = \pNIn,
	no/.estore in = \pNOut,
	nf/.estore in = \pNFunc,
	f1/.estore in = \pFuncA,
	f2/.estore in = \pFuncB,
	f3/.estore in = \pFuncC,
	f4/.estore in = \pFuncD,
	line sep/.estore in = \pLineSep,
	reverse/.is toggle,
}
% Generic applicative leaf process
% #1 = environment keys
% #2 = node name
% #3 = node position
% #4 = node label
\newcommand{\applicative}[4][]{
	\pgfkeys{/applicative, default, #1}%
	\node[inner sep=0pt] (#2_label) at (#3) {\labelStyle{#4}};
	\node[func\pNFunc, font=\funcStyle, yshift=\sepq+\pLineSep, anchor=south] (f#2) at (#2_label.north) {
		\ifnum \pNFunc>0 \nodepart{fa} \pFuncA\else\fi
		\ifnum \pNFunc>1 \nodepart{fb} \pFuncB\else\fi
		\ifnum \pNFunc>2 \nodepart{fc} \pFuncC\else\fi
		\ifnum \pNFunc>3 \nodepart{fd} \pFuncD\else\fi
	};
	\node[anchor=north, yshift=-\sepq-\pLineSep, inner sep=\sepq] (#2_type) at (#2_label.south) {\typeStyle{\pType\ifnolabel\else\getmoclabel{\pMoc}\fi}};
	\iftoggle{/applicative/reverse}
		{\node[i\pNIn o\pNOut, rotate=180, inner sep=0pt, fit=(f#2)(#2_label)(#2_type),] (#2) {};}
		{\node[i\pNIn o\pNOut, inner sep=0pt, fit=(f#2)(#2_label)(#2_type),] (#2) {};}
	\begin{pgfonlayer}{background}
		\node[applshape, draw, moc=\pMoc, inner sep=0pt, fit=(f#2)(#2_label)(#2_type)] {};
	\end{pgfonlayer}
}

% Primitive 
\pgfkeys{
	/primitive/.is family, /primitive,
	default/.style = {moc=none, reverse = false, type= , ni=1, no=1,},
 	moc/.estore in = \pMoc,
 	type/.estore in = \pType,
	ni/.estore in = \pNIn,
	no/.estore in = \pNOut,
	reverse shape/.is toggle,
	reverse/.is toggle,
}
% Generic primite box-shaped leaf process
% #1 = environment keys
% #2 = node name
% #3 = node position
% #4 = node label
\newcommand{\primitivebox}[4][]{
	\applicative[#1, nf=0, line sep=-2pt]{#2}{#3}{#4}
}
% Generic primite special-shaped leaf process
% #1 = environment keys
% #2 = node name
% #3 = node position
\newcommand{\primitivespecial}[3][]{
	\pgfkeys{/primitive, default, #1}%
	\pgfmathsetlength{\foo}{max(\pNIn,\pNOut)*5pt}
	\iftoggle{/primitive/reverse shape}
		{\node[\pType, draw, moc=\pMoc, inner sep=\foo, rotate=180,] (#2_shape) at (#3) {};}	
		{\node[\pType, draw, moc=\pMoc, inner sep=\foo,] (#2_shape) at (#3) {};}	
	\iftoggle{/primitive/reverse}
		{\node[i\pNIn o\pNOut, rotate=180, inner sep=0pt, fit=(#2_shape),] (#2) {};}
		{\node[i\pNIn o\pNOut, inner sep=0pt, fit=(#2_shape),] (#2) {};}
}

% Interface
\pgfkeys{
	/interface/.is family, /interface,
	default/.style = {mocin = none, mocout = none, },
 	mocin/.estore in = \pMocIn,
 	mocout/.estore in = \pMocOut,
	reverse/.is toggle,
}
\newcommand{\interface}[3][]{
	\pgfkeys{/interface, default, #1}%
	\iftoggle{/interface/reverse}{
		\node[domaininterfacerev, mocin=\pMocIn, mocout=\pMocOut, minimum width=30pt, minimum height=30pt, draw] (#2_shape) at (#3) {
			\nodepart{mocin} \typeStyle{\ifnolabel\else\getmoclabel{\pMocIn}\fi}
			\nodepart{mocout} \typeStyle{\ifnolabel\else\getmoclabel{\pMocOut}\fi} 
		};
		\node[i1o1, rotate=180, inner sep=0pt, fit=(#2_shape),] (#2) {};
	}{
		\node[domaininterface, mocin=\pMocIn, mocout=\pMocOut, minimum width=30pt, minimum height=30pt, draw] (#2_shape) at (#3) {
			\nodepart{mocin} \typeStyle{\ifnolabel\else\getmoclabel{\pMocIn}\fi}
			\nodepart{mocout} \typeStyle{\ifnolabel\else\getmoclabel{\pMocOut}\fi} 
		};
		\node[i1o1, inner sep=0pt, fit=(#2_shape),] (#2) {};
	}	
}

%Composite
\pgfkeys{
	/composite/.is family, /composite,
	default/.style = {ni=0, no=0 ,inner sep = 15pt},
 	inner sep/.estore in = \innerSep,
	ni/.estore in = \pNIn,
	no/.estore in = \pNOut,
	reverse/.is toggle,
}
% Generic composite process
% #1 = environment keys
% #2 = node name
% #3 = list of nodes clustered
% #4 = node label
\newcommand\composite[4][]{
	\pgfkeys{/composite, default, #1}%
	\node[composite, inner sep=\innerSep, fit=#3 , draw] (#2_shape) {};
	\node[anchor=south east] (#2_label) at (#2_shape.south east) {\footnotesize\labelStyle{#4}}; 
	\iftoggle{/composite/reverse}
		{\node[i\pNIn o\pNOut, rotate=180, inner sep=0pt, fit=(#2_shape),] (#2) {};}
		{\node[i\pNIn o\pNOut, inner sep=0pt, fit=(#2_shape),] (#2) {};}
}
% Generic black-box composite process
% #1 = environment keys
% #2 = node name
% #3 = node position
% #4 = node label
\newcommand\compositebbox[4][]{
	\pgfkeys{/composite, default, #1}%
	\node[rectangle, rounded corners = 3pt, minimum width=30pt, minimum height=30pt, inner sep=\innerSep,
		draw=\defaultdrawcolor, fill=blackboxcolor] (#2_shape) at (#3) {\labelStyle{#4}};
	\iftoggle{/composite/reverse}
		{\node[i\pNIn o\pNOut, rotate=180, inner sep=0pt, fit=(#2_shape),] (#2) {};}
		{\node[i\pNIn o\pNOut, inner sep=0pt, fit=(#2_shape),] (#2) {};}
}

%Parallel computation
\pgfkeys{%
	/parcomp/.is family, /parcomp,
	default/.style = {reverse = false, reverse shape=false, type=genericpar , ni=1, no=1, nf=0, f1=$ \langle f_1\rangle $, f2=$ \langle f_2\rangle $, f3=$ \langle f_3\rangle $, f4=$ \langle f_4\rangle $, inner sep=18pt},
 	type/.estore in = \pType,
	ni/.estore in = \pNIn,
	no/.estore in = \pNOut,
	nf/.estore in = \pNFunc,
	f1/.estore in = \pFuncA,
	f2/.estore in = \pFuncB,
	f3/.estore in = \pFuncC,
	f4/.estore in = \pFuncD,
	inner sep/.estore in = \innerSep,
	reverse/.is toggle,
	reverse shape/.is toggle,
}
% #1 = environment keys
% #2 = node name
% #3 = list of nodes clustered
% #4 = node label
\newcommand\parcomp[4][]{
	\pgfkeys{/parcomp, default, #1}%
	\iftoggle{/parcomp/reverse}{
		\iftoggle{/parcomp/reverse shape}
		 {\node[\pType, draw, inner sep=\innerSep, fit=#3,] (#2_shape) {};}
		 {\node[\pType, draw, inner sep=\innerSep, fit=#3, rotate=180] (#2_shape) {};}
		\node[i\pNIn o\pNOut, rotate=180, inner sep=0pt, fit=(#2_shape),] (#2) {};
		\node[anchor=south east] (#2_label) at (#2_shape.north west) {\footnotesize\labelStyle{#4}}; 
		\node[func\pNFunc, font=\funcStyle, anchor=south west] (f#2) at (#2_shape.south east) {
			\ifnum \pNFunc>0 \nodepart{fa} \pFuncA\else\fi
			\ifnum \pNFunc>1 \nodepart{fb} \pFuncB\else\fi
			\ifnum \pNFunc>2 \nodepart{fc} \pFuncC\else\fi
			\ifnum \pNFunc>3 \nodepart{fd} \pFuncD\else\fi
		};
	}{
		\iftoggle{/parcomp/reverse shape}
		 {\node[\pType, draw, inner sep=\innerSep, fit=#3,rotate=180] (#2_shape) {};}
		 {\node[\pType, draw, inner sep=\innerSep, fit=#3] (#2_shape) {};}
		\node[i\pNIn o\pNOut, inner sep=0pt, fit=(#2_shape),] (#2) {};
		\node[anchor=south east] (#2_label) at (#2_shape.south east) {\footnotesize\labelStyle{#4}}; 
		\node[func\pNFunc, font=\funcStyle, anchor=south west] (f#2) at (#2_shape.north west) {
			\ifnum \pNFunc>0 \nodepart{fa} \pFuncA\else\fi
			\ifnum \pNFunc>1 \nodepart{fb} \pFuncB\else\fi
			\ifnum \pNFunc>2 \nodepart{fc} \pFuncC\else\fi
			\ifnum \pNFunc>3 \nodepart{fd} \pFuncD\else\fi
		};
	}
}

% Parallel communication
\pgfkeys{%
	/parcomm/.is family, /parcomm,
	default/.style = {reverse = false, inner type=noinner, outer type=vvbase , 	reverse inner= false, 	reverse outer=false, ni=1, no=1, nf=0, f1=$f_1$, f2=$f_2$, f3=$f_3$, f4=$f_4$,},
 	inner type/.estore in = \pInnerType,
 	outer type/.estore in = \pOuterType,
	ni/.estore in = \pNIn,
	no/.estore in = \pNOut,
	nf/.estore in = \pNFunc,
	f1/.estore in = \pFuncA,
	f2/.estore in = \pFuncB,
	f3/.estore in = \pFuncC,
	f4/.estore in = \pFuncD,
	reverse/.is toggle,
	reverse inner/.is toggle,
	reverse outer/.is toggle,
}
\tikzset{% 
    noinner/.style={draw=none, inner sep=0pt, minimum height=30pt, minimum width=3pt},
    funcbox/.style={draw=none, fill=\defaultfillcolor, font=\scriptsize, rotate=90, minimum width=30pt, minimum height=7pt},
    invbox/.style={draw=none, fill=\defaultfillcolor, rotate=180, minimum height=30pt, minimum width=7pt},
}
% Generic visually represented parallel communication process
% #1 = environment keys
% #2 = node name
% #3 = node position
\newcommand\parcomm[3][]{
	\pgfkeys{/parcomm, default, #1}%
	\iftoggle{/parcomm/reverse}{
		\iftoggle{/parcomm/reverse inner}
			{\node[\pInnerType] (v#2) at (#3) {};}
			{\node[\pInnerType, rotate=180] (v#2) at (#3) {};}
		\iftoggle{/parcomm/reverse outer}
			{\node[\pOuterType, draw, inner sep=0pt, fit=(v#2),] (#2_outershape) {};}
			{\node[\pOuterType, draw, inner sep=0pt, fit=(v#2), rotate=180,] (#2_outershape) {};}
		\node[i\pNIn o\pNOut, rotate=180, inner sep=0pt, fit=(#2_outershape),] (#2) {};
	}{
		\iftoggle{/parcomm/reverse inner}
			{\node[\pInnerType, rotate=180] (v#2) at (#3) {};}
			{\node[\pInnerType] (v#2) at (#3) {};}
		\iftoggle{/parcomm/reverse outer}
			{\node[\pOuterType, draw, inner sep=0pt, fit=(v#2), rotate=180,] (#2_outershape) {};}
			{\node[\pOuterType, draw, inner sep=0pt, fit=(v#2),] (#2_outershape) {};}
		\node[i\pNIn o\pNOut, inner sep=0pt, fit=(#2_outershape),] (#2) {};
	}
}
% Generic function-based parallel communication process
% #1 = environment keys
% #2 = node name
% #3 = node position
% #4 = function
\newcommand\parcommfunc[4][]{
	\pgfkeys{/parcomm, default, #1}%
	\node[funcbox] (v#2) at (#3) { #4};
	\node[func\pNFunc, font=\funcStyle, anchor=south] (f#2) at (v#2.east) {
		\ifnum \pNFunc>0 \nodepart{fa} \pFuncA\else\fi
		\ifnum \pNFunc>1 \nodepart{fb} \pFuncB\else\fi
		\ifnum \pNFunc>2 \nodepart{fc} \pFuncC\else\fi
		\ifnum \pNFunc>3 \nodepart{fd} \pFuncD\else\fi
	};
	\iftoggle{/parcomm/reverse}{
		\iftoggle{/parcomm/reverse outer}
			{\node[\pOuterType, draw, inner sep=1pt, fit=(v#2)(f#2)] (#2_outershape) {};}
			{\node[\pOuterType, draw, inner sep=1pt, fit=(v#2)(f#2), rotate=180] (#2_outershape) {};}
		\node[i\pNIn o\pNOut, rotate=180, inner sep=0pt, fit=(#2_outershape),] (#2) {};
	}{
		\iftoggle{/parcomm/reverse outer}
			{\node[\pOuterType, draw, inner sep=1pt, fit=(v#2)(f#2), rotate=180] (#2_outershape) {};}
			{\node[\pOuterType, draw, inner sep=1pt, fit=(v#2)(f#2)] (#2_outershape) {};}
			\node[i\pNIn o\pNOut, inner sep=0pt, fit=(#2_outershape),] (#2) {};
	}
}
\newcommand\parcommbox[4][]{
	\pgfkeys{/parcomm, default, #1}%
	\node[\pInnerType] (v#2) at (#3) {#4};
	\iftoggle{/parcomm/reverse}{
		\iftoggle{/parcomm/reverse outer}
			{\node[\pOuterType, draw, inner sep=1pt, fit=(v#2)] (#2_outershape) {};}
			{\node[\pOuterType, draw, inner sep=1pt, fit=(v#2), rotate=180] (#2_outershape) {};}
		\node[i\pNIn o\pNOut, rotate=180, inner sep=0pt, fit=(#2_outershape),] (#2) {};
	}{
		\iftoggle{/parcomm/reverse outer}
			{\node[\pOuterType, draw, inner sep=1pt, fit=(v#2), rotate=180] (#2_outershape) {};}
			{\node[\pOuterType, draw, inner sep=1pt, fit=(v#2)] (#2_outershape) {};}									\node[i\pNIn o\pNOut, inner sep=0pt, fit=(#2_outershape),] (#2) {};
	}
}

%%%%%%%%%
% PATHS %
%%%%%%%%%
\def\signal[#1] (#2) #3 (#4);{
        \draw (#2) edge[#1, #3, s=\MoC,] (#4);
}
\def\vector[#1] (#2) #3 (#4);{
        \draw (#2) edge[#1, #3, v=\MoC,] (#4);
}
\def\function[#1] (#2) #3 (#4);{
        \draw (#2) edge[#1, #3, f,] (#4);
}
