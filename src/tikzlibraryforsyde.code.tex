\newcommand{\getmoclabel}[1]{\ifthenelse{\equal{#1}{sy}}{SY}{\ifthenelse{\equal{#1}{de}}{DE}{\ifthenelse{\equal{#1}{ct}}{CT}{\ifthenelse{\equal{#1}{sdf}}{SDF}{}}}}}
\newcommand{\functionlabels}[5]{ %#1=nf, #2... = functions
	\ifnum #1>0 \nodepart{fa} #2\else\fi
	\ifnum #1>1 \nodepart{fb} #3\else\fi
	\ifnum #1>2 \nodepart{fc} #4\else\fi
	\ifnum #1>3 \nodepart{fd} #5\else\fi
}

%%%%%%%%%%%%%%%%%%
% LEAF PROCESSES %
%%%%%%%%%%%%%%%%%%

% Applicative

% Standard leaf process primitive
% #1 = keys
% #2 = name
% #3 = position
% #4 = label
\newcommand{\leafstd}[4][]{
	\pgfkeys{/standard leaf keys, default, #1}%
	\pgfmathtruncatemacro\rot{\iftoggle{/standard leaf keys/reverse}{180}{0}}
	\begin{scope}[local bounding box= #2, shift={(#3)}]
		\node[] 
			(#2-labl) at (0,0)           {\labelStyle{#4}};
		\node[func\pNFunc, font=\funcStyle, yshift=\pInnerSep, anchor=north] 
			(#2-f)    at (#2-labl.north)  {\functionlabels{\pNFunc}{\pFuncA}{\pFuncB}{\pFuncC}{\pFuncD}};
		\node[anchor=south, yshift=-\pInnerSep, inner sep=2pt] 
			(#2-type) at (#2-labl.south) {\typeStyle{\pType\ifnolabel\else\getmoclabel{\pMoc}\fi}};
		\node[ports i\pNIn o\pNOut, rotate=\rot, inner sep=0pt, fit=(#2-f)(#2-labl)(#2-type),] (#2-p) {};
		\begin{pgfonlayer}{background}
			\node[std leaf shape, draw, moc=\pMoc, inner sep=0pt, fit=(#2-f)(#2-labl)(#2-type)] {};
		\end{pgfonlayer}
	\end{scope}
}

% Custom leaf process primitive
% #1 = keys
% #2 = name
% #3 = position
\newcommand{\leafcustom}[3][]{
	\pgfkeys{/standard leaf keys, default, #1}%
	\pgfmathsetlength{\foo}{max(max(\pNIn,\pNOut)*5pt, \pInnerSep)}
	\pgfmathtruncatemacro\rotports{\iftoggle{/standard leaf keys/reverse}{180}{0}}
	\pgfmathtruncatemacro\rotshape{\iftoggle{/standard leaf keys/reverse shape}{180}{0}}
	\begin{scope}[local bounding box= #2, shift={(#3)}]
		\node[\pShape, draw, moc=\pMoc, inner sep=\foo, rotate=\rotshape,] (#2-shape) at (0,0) {};
		\node[ports i\pNIn o\pNOut, rotate=\rotports, inner sep=0pt, fit=(#2-shape),] (#2-p) {};
	\end{scope}
}

% Interface
\newcommand{\interface}[3][]{
	\pgfkeys{/interface keys, default, #1}%
	\pgfmathtruncatemacro\rot{\iftoggle{/interface keys/reverse}{180}{0}}	
	\begin{scope}[local bounding box= #2, shift={(#3)}]
		\node[interface shape \rot, mocin=\pMocIn, mocout=\pMocOut, minimum width=30pt, minimum height=30pt, draw] (#2-shape) at (0,0) {
			\nodepart{mocin} \typeStyle{\ifnolabel\else\getmoclabel{\pMocIn}\fi}
			\nodepart{mocout} \typeStyle{\ifnolabel\else\getmoclabel{\pMocOut}\fi} 
		};
		\node[ports i1o1, rotate=\rot, inner sep=0pt, fit=(#2-shape),] (#2-p) {};
	\end{scope}
}

%%%%%%%%%%%%%%%%%%%%%%%
% COMPOSITE PROCESSES %
%%%%%%%%%%%%%%%%%%%%%%%

%Composite

% Generic composite process
% #1 = keys
% #2 = name
% #3 = list of nodes clustered
% #4 = label
\newcommand\compositestd[4][]{
	\pgfkeys{/composite keys, default, #1}%
	\pgfmathtruncatemacro\rot{\iftoggle{/composite keys/reverse}{180}{0}}
	\begin{scope}[local bounding box= #2]
		\node[composite shape, inner xsep=\innerXSep, inner ysep=\innerYSep, fit=#3 , draw] (#2-shape) {};
		\node[anchor=south east] (#2-label) at (#2-shape.south east) {\footnotesize\labelStyle{#4}}; 
		\node[ports i\pNIn o\pNOut, rotate=\rot, inner sep=0pt, fit=(#2-shape),] (#2-p) {};
	\end{scope}
}
% Generic black-box composite process
% #1 = keys
% #2 = name
% #3 = position
% #4 = label
\newcommand\compositebbox[4][]{
	\pgfkeys{/composite keys, default, #1}%
	\pgfmathtruncatemacro\rot{\iftoggle{/composite/reverse}{180}{0}}
	\begin{scope}[local bounding box= #2]
		\node[rectangle, rounded corners = 3pt, minimum width=30pt, minimum height=30pt, inner sep=\innerSep,
			draw=\defaultdrawcolor, fill=blackboxcolor] (#2-shape) at (#3) {\labelStyle{#4}};
		\node[ports i\pNIn o\pNOut, rotate=\rot, inner sep=0pt, fit=(#2-shape),] (#2-p) {};
	\end{scope}
}


%%%%%%%%%%%%
% PATTERNS %
%%%%%%%%%%%%

% Applicative pattern
% #1 = keys
% #2 = name
% #3 = list of nodes clustered
% #4 = label
\newcommand\patternappl[4][]{
	\pgfkeys{/applicative patterns keys, default, #1}%
	\pgfmathtruncatemacro\rotports{\iftoggle{/applicative patterns keys/reverse}{180}{0}}
	\pgfmathtruncatemacro\rotshape{\iftoggle{/applicative patterns keys/reverse shape}
		{\iftoggle{/applicative patterns keys/reverse}{0}{180}}
		{\iftoggle{/applicative patterns keys/reverse}{180}{0}}}
	\begin{scope}[local bounding box= #2]
		\node[\pShape, draw, inner xsep=\innerXSep, inner ysep=\innerYSep, fit=#3, rotate=\rotshape]
			(#2-shape) {};
		\node[inner xsep=\innerXSep, inner ysep=\innerYSep, fit=#3] (#2-scope) {};
		\node[ports i\pNIn o\pNOut, rotate=\rotports, inner sep=0pt, fit=(#2-shape),]  (#2-p) {};	
		\node[anchor=south east] (#2-label) at (#2-scope.south east) {\footnotesize\labelStyle{#4}}; 
		\node[anchor=south west] (#2-type)  at (#2-scope.south west) {\typeStyle{\pType}}; 
		\node[func\pNFunc, font=\funcStyle, anchor=south west] (#2-f) at (#2-scope.north west) 
			{\functionlabels{\pNFunc}{\pFuncA}{\pFuncB}{\pFuncC}{\pFuncD}};
	\end{scope}
}
% Standard transition pattern
% #1 = keys
% #2 = name
% #3 = position
\newcommand\patterntransstd[3][]{
	\pgfkeys{/trasition patterns keys, default, #1}%
	\pgfmathtruncatemacro\rotports{\iftoggle{/trasition patterns keys/reverse}{180}{0}}
	\begin{scope}[local bounding box= #2, shift={(#3)}]	
		\begin{scope}	
			\node[rotate=90] (#2-type) at (0,0) {\typeStyle{\pType}};
			\node[fit=(#2-type), inner sep=0pt](#2-wt) {};
			\node[func\pNFunc, font=\funcStyle, anchor=south, yshift=2pt] (#2-f) at (#2-wt.north) 
				{\functionlabels{\pNFunc}{\pFuncA}{\pFuncB}{\pFuncC}{\pFuncD}};
			\node[\pOuterShape, draw, inner sep=0pt, fit=(#2-f)(#2-wt), rotate=\rotports,] (#2-shape) {};
			\node[ports i\pNIn o\pNOut, rotate=\rotports, inner sep=0pt, fit=(#2-shape),] (#2-p) {};
		\end{scope}
	\end{scope}
}
% Custom transition pattern
% #1 = keys
% #2 = name
% #3 = position
\newcommand\patterntranscustom[3][]{
	\pgfkeys{/trasition patterns keys, default, #1}%
	\pgfmathtruncatemacro\rotports{\iftoggle{/trasition patterns keys/reverse}{180}{0}}
	\pgfmathtruncatemacro\rotshape{\iftoggle{/trasition patterns keys/reverse shape}
		{\iftoggle{/trasition patterns keys/reverse}{0}{180}}
		{\iftoggle{/trasition patterns keys/reverse}{180}{0}}}
	\begin{scope}[local bounding box= #2, shift={(#3)}]	
		\begin{scope}	
			\node[\pInnerShape, rotate=\rotports] (#2-c) at (0,0) {};
			\node[func\pNFunc, font=\funcStyle, anchor=south] (#2-f) at (#2-c.north) 
				{\functionlabels{\pNFunc}{\pFuncA}{\pFuncB}{\pFuncC}{\pFuncD}};
			\node[\pOuterShape, draw, inner xsep=0pt, inner ysep=1pt, fit=(#2-c), rotate=\rotshape,] 
				(#2-oshape) {};
			\node[ports i\pNIn o\pNOut, rotate=\rotports, inner sep=0pt, fit=(#2-oshape)(#2-f),] (#2-p) {};
		\end{scope}
	\end{scope}
}

%%%%%%%%%
% PATHS %
%%%%%%%%%
\def\signal[#1] (#2) #3 (#4);{
        \draw (#2) edge[#1, #3, s=\MoC,] (#4);
}
\def\vector[#1] (#2) #3 (#4);{
        \draw (#2) edge[#1, #3, v=\MoC,] (#4);
}
\def\function[#1] (#2) #3 (#4);{
        \draw (#2) edge[#1, #3, f,] (#4);
}
