% pgf/tikz library
% for ForSyDe signals
%
% Author: George Ungureanu, KTH - Royal Institute of Technology, Sweden
% Version: 0.3
% Date: 2015/05/17

%
% Extended general shape options, cf. tikz.code.tex
%
% #1 = source node
% Remark: The node distance is used
%         for the distance between the borders of two nodes
\tikzoption{below from}{\tikz@from{#1}{1}{-90}{south}{north}}%
\tikzoption{right from}{\tikz@from{#1}{1}{0}{east}{west}}%
\tikzoption{above from}{\tikz@from{#1}{1}{90}{north}{south}}%
\tikzoption{left from}{\tikz@from{#1}{1}{180}{west}{east}}%
\tikzoption{below left from}{\tikz@from{#1}{1.414214}{-135}{south west}{north east}}%
\tikzoption{below right from}{\tikz@from{#1}{1.414214}{-45}{south east}{north west}}%
\tikzoption{above right from}{\tikz@from{#1}{1.414214}{45}{north east}{south west}}%
\tikzoption{above left from}{\tikz@from{#1}{1.414214}{135}{north west}{south east}}%
\def\tikz@from#1#2#3#4#5{%
  \def\tikz@anchor{#5}%
  \let\tikz@do@auto@anchor=\relax%
  \tikz@addtransform{\pgftransformshift{\pgfpointscale{#2}{\pgfpointpolar{#3}{\tikz@node@distance}}}}%
  \def\tikz@node@at{\pgfpointanchor{#1}{#4}}}


% Styles for signal, vector and function tips.
\newcommand{\pgfarrowsextend}[1]{%
  \ifdim \pgflinewidth>2pt \@tempdima=\vectorportsize 
  \else                    \@tempdima=\signalportsize 
  \fi
  \pgfarrowsleftextend{-#1\@tempdima}
  \pgfarrowsrightextend{#1\@tempdima}
}
\newcommand{\pgfarrowsatip}[1]{% dimension, proportion1, proportion2
  \ifdim \pgflinewidth>2pt \@tempdima=\vectorportsize 
  \else                    \@tempdima=\signalportsize 
  \fi
  \pgfpathmoveto{\pgfpoint{2\@tempdima}{0pt}}
  \pgfpathlineto{\pgfpoint{0}{\@tempdima}}
  \pgfpathlineto{\pgfpointorigin}
  \pgfpathlineto{\pgfpoint{0}{-\@tempdima}}
  \pgfusepathqfill
}
\newcommand{\pgfarrowsatipsq}[1]{%
  \ifdim \pgflinewidth>2pt \@tempdima=\vectorportsize 
  \else                    \@tempdima=\signalportsize 
  \fi
  \pgfpathmoveto{\pgfpoint{2\@tempdima}{0pt}}
  \pgfpathlineto{\pgfpoint{0}{\@tempdima}}
  \pgfpathlineto{\pgfpointorigin}
  \pgfpathlineto{\pgfpoint{0}{-\@tempdima}}
  \pgfusepathqfill
  \pgfpathmoveto{\pgfpoint{4\@tempdima}{-\@tempdima}}
  \pgfpathlineto{\pgfpoint{4\@tempdima}{\@tempdima}}
  \pgfpathlineto{\pgfpoint{2\@tempdima}{\@tempdima}}
  \pgfpathlineto{\pgfpoint{2\@tempdima}{-\@tempdima}}
  \pgfpathclose
  \pgfsetfillcolor{\defaultdrawcolor}
  \pgfusepathqfill
}
\newcommand{\pgfarrowsasq}[1]{%
  \ifdim \pgflinewidth>2pt \@tempdima=\vectorportsize 
  \else                    \@tempdima=\signalportsize 
  \fi
  \pgfpathmoveto{\pgfpoint{\@tempdima}{-\@tempdima}}
  \pgfpathlineto{\pgfpoint{\@tempdima}{\@tempdima}}
  \pgfpathlineto{\pgfpoint{-\@tempdima}{\@tempdima}}
  \pgfpathlineto{\pgfpoint{-\@tempdima}{-\@tempdima}}
  \pgfpathlineto{\pgfpoint{\@tempdima}{-\@tempdima}}
  \pgfsetfillcolor{\defaultdrawcolor}
  \pgfusepathqfill
}
\newcommand{\pgfarrowsbtipsq}[1]{%
  \pgfsetdash{{1cm}}{0cm} 
  \pgfsetcolor{\defaultdrawcolor}%
  \@tempdima=#1\pgflinewidth%
  \pgfpathmoveto{\pgfpoint{\@tempdima}{-\@tempdima}}
  \pgfpathlineto{\pgfpoint{\@tempdima}{\@tempdima}}
  \pgfpathlineto{\pgfpoint{-\@tempdima}{\@tempdima}}
  \pgfpathlineto{\pgfpoint{-\@tempdima}{-\@tempdima}}
  \pgfpathclose
  \pgfsetfillcolor{\defaultfillcolor}
  \pgfusepathqfill
  \pgfpathmoveto{\pgfpoint{\@tempdima}{-\@tempdima}}
  \pgfpathlineto{\pgfpoint{\@tempdima}{\@tempdima}}
  \pgfpathlineto{\pgfpoint{-\@tempdima}{\@tempdima}}
  \pgfpathlineto{\pgfpoint{-\@tempdima}{-\@tempdima}}
  \pgfpathclose
  \pgfsetstrokecolor{\defaultdrawcolor}
  \pgfusepathqstroke
}
\newcommand{\pgfarrowsbsq}[1]{%
  \pgfsetdash{{1cm}}{0cm} 
  \pgfsetcolor{\defaultdrawcolor}%
  \@tempdima=#1\pgflinewidth%
  \pgfpathmoveto{\pgfpoint{2\@tempdima}{0pt}}
  \pgfpathlineto{\pgfpoint{0}{\@tempdima}}
  \pgfpathlineto{\pgfpointorigin}
  \pgfpathlineto{\pgfpoint{0}{-\@tempdima}}
  \pgfpathlineto{\pgfpoint{2\@tempdima}{0pt}}
  \pgfpathclose
  \pgfusepathqstroke
}

\pgfarrowsdeclare{stip}{stip}{\pgfarrowsextend{2}}{\pgfarrowsatip{2}}
\pgfarrowsdeclare{>p}{>p}    {\pgfarrowsextend{3}}{\pgfarrowsatipsq{2}}
\pgfarrowsdeclare{p<}{p<}    {\pgfarrowsextend{3}}{\pgfarrowsatipsq{2}}
\pgfarrowsdeclare{p}{p}      {\pgfarrowsextend{0}}{\pgfarrowsasq{2}}
\pgfarrowsdeclare{ftip}{ftip}{\pgfarrowsextend{1}}{\pgfarrowsbtipsq{3}}
\pgfarrowsdeclare{fi}{fi}    {\pgfarrowsextend{-1.5}}{\pgfarrowsbsq{3}}


%
% Styles for signal, vector and function paths.
%
\tikzset{
  getsignalmoc/.code={
	\ifnocolor
		\defaultdrawcolor
	\else
	  	\ifthenelse{ \equal{#1}{sy}}{\tikzset{sycolor}}{
	  	\ifthenelse{ \equal{#1}{de}}{\tikzset{decolor}}{
	  	\ifthenelse{ \equal{#1}{ct}}{\tikzset{ctcolor}}{
	  	\ifthenelse{ \equal{#1}{sdf}}{\tikzset{sdfcolor}}{}
	  	}}}
    \fi
  },
  s/.style args  = {#1}{getsignalmoc={#1}, line width=\signalpathlinewidth, >= stip, draw},
  v/.style args  = {#1}{getsignalmoc={#1}, line width=\vectorpathlinewidth, >= stip, draw},
  f/.style       = {line width=\functionpathlinewidth, >= ftip, dotted, shorten >= 1mm, shorten <= 1mm, draw},
}

%
% Styles for data tokens.
%
\newlength{\xTemp}
\newlength{\inTemp}
\ExplSyntaxOn

\NewDocumentCommand{\tokens}{m}
 {% #1 = character, #2 = string
  \youra_count_char:nn { - } { #1 }
  \pgfmathsetlength{\xTemp}{-\l_youra_count_char_int * \tokensize}
  \setlength{\inTemp}{\halftokensize}
  \drawtokens{ #1 }
 }
\int_new:N \l_youra_count_char_int
\cs_new_protected:Npn \youra_count_char:nn #1 #2
 {
  \regex_count:nnN { #1 } { #2 } \l_youra_count_char_int
 }
 
\NewDocumentCommand{\drawtokens}{ >{ \SplitList { - } } m }
 {
  \ProcessList { #1 } { \davs__tokens_rec:n }
 }
\cs_new_protected:Nn \davs__tokens_rec:n
 {
  \davs__tokens_do:n { #1 } 
  
  \pgfmathsetlength{\xTemp}{\xTemp+\tokensize+\tokensize}
 }
\cs_new_protected:Nn \davs__tokens_do:n
 {
	\ifthenelse{ \equal{#1}{scalar}}{
		\draw[fill=\defaultfillcolor, thin] (\xTemp,0) circle [radius=\tokensize];
	}{
  	\ifthenelse{ \equal{#1}{vector}}{
		\draw[fill=\defaultfillcolor, thin] (\xTemp-\tokensize,0) -- (\xTemp,-\tokensize) -- (\xTemp+\tokensize,0) -- (\xTemp,\tokensize) -- cycle ;
	}{
  	\ifthenelse{ \equal{#1}{function}}{
		\draw[fill=\defaultfillcolor, thin ] (\xTemp-\tokensize,-\tokensize) rectangle (\xTemp+\tokensize,\tokensize);
	}{
  	\ifthenelse{ \equal{#1}{scalarAE}}{
		\draw[fill=\defaultfillcolor, thin] (\xTemp,0) circle [radius=\tokensize];
		\pgfsetlinewidth{.5pt} 
		\pgfpathmoveto{\pgfpoint{\xTemp}{-.75\inTemp}}
		\pgfpathlineto{\pgfpoint{\xTemp}{\inTemp}}
		\pgfusepathqstroke
		\pgfpathmoveto{\pgfpoint{\xTemp-\inTemp}{-.75\inTemp}}
		\pgfpathlineto{\pgfpoint{\xTemp+\inTemp}{-.75\inTemp}}
		\pgfusepathqstroke
	}{
	\ifthenelse{ \equal{#1}{vectorAE}}{
		\draw[fill=\defaultfillcolor, thin ] (\xTemp-\tokensize,0) -- (\xTemp,-\tokensize) -- (\xTemp+\tokensize,0) -- (\xTemp,\tokensize) -- cycle;
		\pgfsetlinewidth{.5pt} 
		\pgfpathmoveto{\pgfpoint{\xTemp}{-.75\inTemp}}
		\pgfpathlineto{\pgfpoint{\xTemp}{\inTemp}}
		\pgfusepathqstroke
		\pgfpathmoveto{\pgfpoint{\xTemp-.75\inTemp}{-.75\inTemp}}
		\pgfpathlineto{\pgfpoint{\xTemp+.75\inTemp}{-.75\inTemp}}
		\pgfusepathqstroke
	}{
  	\ifthenelse{ \equal{#1}{functionAE}}{
		\draw[fill=\defaultfillcolor, thin ] (\xTemp-\tokensize,-\tokensize) rectangle (\xTemp+\tokensize,\tokensize);
		\pgfsetlinewidth{.5pt} 
		\pgfpathmoveto{\pgfpoint{\xTemp}{-\inTemp}}
		\pgfpathlineto{\pgfpoint{\xTemp}{\inTemp}}
		\pgfusepathqstroke
		\pgfpathmoveto{\pgfpoint{\xTemp-\inTemp}{-\inTemp}}
		\pgfpathlineto{\pgfpoint{\xTemp+\inTemp}{-\inTemp}}
		\pgfusepathqstroke
	}{}
	}}}}}
 }
\ExplSyntaxOff


%
% Path splines syntax
%
\tikzset{
  token/.style={decoration={name=markings, post length=1pt, pre length=1pt, mark=at position \tokPos with {\tokens{#1}}}, postaction=decorate },
  -|-/.style={ to path={ (\tikztostart) 
	-| ([xshift=\pDeviate]$(\tikztostart)!#1!(\tikztotarget)$) 
	|- (\tikztotarget) 
	\tikztonodes }
  }, -|-/.default=0.5,
  |-|/.style={to path={ (\tikztostart) 
	|- ($(\tikztostart)!#1!(\tikztotarget)$) 
	-| (\tikztotarget)
    \tikztonodes }
  }, |-|/.default=0.5,
  -|-|/.style args = {#1}{ to path={ (\tikztostart) 
	-| ($(\tikztostart)!#1!(\tikztotarget)$) 
	|- ([yshift=\pDeviate]\tikztotarget) 
	-| (\tikztotarget) 
	\tikztonodes }
  }, -|-|/.default=0.5,
  -|-|-/.style args = {#1:#2}{ to path={ (\tikztostart) 
	-| ($(\tikztostart)!#1!(\tikztotarget)$) 
	|- ([yshift=\pDeviate]$(\tikztostart)!#2!(\tikztotarget)$) 
	-| ([yshift=\pDeviate]$(\tikztostart)!#2!(\tikztotarget)$)
	|- (\tikztotarget) 
	\tikztonodes }
  }, -|-|-/.default=0.3:0.7,
  |-|-|/.style args = {#1:#2}{ to path={ (\tikztostart) 
	|- ($(\tikztostart)!#1!(\tikztotarget)$) 
	-| ([xshift=\pDeviate]$(\tikztostart)!#2!(\tikztotarget)$) 
	|- ([xshift=\pDeviate]$(\tikztostart)!#2!(\tikztotarget)$)
	-| (\tikztotarget) 
	\tikztonodes }
  }, |-|-|/.default=0.3:0.7,
}

% Finally, edge drawing helpers
\def\signal[#1] (#2) #3 (#4);{
        \draw (#2) edge[#1, #3, s=\MoC,] (#4);
}
\def\vector[#1] (#2) #3 (#4);{
        \draw (#2) edge[#1, #3, v=\MoC,] (#4);
}
\def\function[#1] (#2) #3 (#4);{
        \draw (#2) edge[#1, #3, f,] (#4);
}

\endinput
